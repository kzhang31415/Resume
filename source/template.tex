%%%%%%%%%%%%%%%%%%%%%%%%%%%%%%%%%%%%%%%%%
% Medium Length Professional CV
% LaTeX Template
% Version 3.0 (December 17, 2022)
%
% This template originates from:
% https://www.LaTeXTemplates.com
%
% Author:
% Vel (vel@latextemplates.com)
%
% Original author:
% Trey Hunner (http://www.treyhunner.com/)
%
% License:
% CC BY-NC-SA 4.0 (https://creativecommons.org/licenses/by-nc-sa/4.0/)
%
%%%%%%%%%%%%%%%%%%%%%%%%%%%%%%%%%%%%%%%%%

%----------------------------------------------------------------------------------------
%	PACKAGES AND OTHER DOCUMENT CONFIGURATIONS
%----------------------------------------------------------------------------------------

\documentclass[
	%a4paper, % Uncomment for A4 paper size (default is US letter)
	11pt, % Default font size, can use 10pt, 11pt or 12pt
]{resume} % Use the resume class

\usepackage{ebgaramond} % Use the EB Garamond font

%------------------------------------------------

\name{Kevin Zhang} % Your name to appear at the top

% You can use the \address command up to 3 times for 3 different addresses or pieces of contact information
% Any new lines (\\) you use in the \address commands will be converted to symbols, so each address will appear as a single line.

 \address{648 East Constitution Drive \\ Palatine, IL 60074} % Main address

\address{(224)-358-9741 \\ kzhang31415@cmu.edu \\ linkedin.com/in/kzhang31415 \\ github.com/kzhang31415} % Contact information

%----------------------------------------------------------------------------------------

\begin{document}

%----------------------------------------------------------------------------------------
%	EDUCATION SECTION
%----------------------------------------------------------------------------------------

\begin{rSection}{Education}
	
	\textbf{Carnegie Mellon University} \hfill \textit{June 2023 - Present} \\ 
	Bachelor of Science, Computer Science \smallskip \\
    Relevant Coursework: Principles of Imperative Computation, Mathematical Foundations for Computer Science, Great Practical Ideas in Computer Science, Matrix Theory \smallskip
 
	\textbf{Illinois Mathematics and Science Academy} \hfill \textit{August 2020 - June 2023} \\ 
	High School Diploma - GPA: 4.0/4.0 \smallskip \\
    Relevant Coursework: Computational Science, CS Seminar: Machine Learning, iOS App Development
	
\end{rSection}

%----------------------------------------------------------------------------------------
%	SKILLS SECTION
%----------------------------------------------------------------------------------------

\begin{rSection}{Skills}
	\begin{tabular}{@{} >{\bfseries}l @{\hspace{6ex}} l @{}}
		Languages & C++, Java, C\#, C, JavaScript, Swift, Python, HTML, CSS, \LaTeX \\
		Technologies & Node, Express, JQuery, Linux, JSON\\
        Skills & Computational Physics, Data Analysis, Machine Learning, Mathematics, Tutoring
	\end{tabular}
\end{rSection}

%----------------------------------------------------------------------------------------
%	EXPERIENCE SECTION
%----------------------------------------------------------------------------------------

\begin{rSection}{Experience}

	\begin{rSubsection}{Fermilab}{June 2021 - June 2023}{Student Researcher}{Batavia, IL}
		\item Lowered lepton jet misclassification rates by 61.8\% from the theoretical limit of cuts-based methods currently used in my group
        \item Validated the invariance of the MLP model against dark photon mass, final state radiation, and event type except those in which dark photon parent particle mass is changed
        \item Deployed machine learning models to classify high same-sign invariant mass ($\geq$ 100 GeV) lepton pairs as signal-generated or background-generated
	\end{rSubsection}

    \begin{rSubsection}{Illinois Mathematics and Science Academy}{August 2021 - June 2023}{Peer Tutor}{Aurora, IL}
        \item Demystified math, physics, and computer science concepts for other students
        \item Accumulated over 250 hours of tutoring experience
    \end{rSubsection}

%------------------------------------------------

\end{rSection}

\begin{rSection}{Projects}

%----------------------------------------------------------------------------------------
%	PROJECTS SECTION
%----------------------------------------------------------------------------------------

    \begin{rSubsection}{IMSA Courses}{December 2020 - April 2021}{}{}
		\item A website which compiles information on course offerings at my high school, IMSA, into a more efficient and navigable format. This information was sourced from the Course Catalog found at https://imsa.edu/academics/academic-programs/. Information can be updated to remain accurate over time by teachers and administrators directly from the website.
	\end{rSubsection}
\end{rSection}

%----------------------------------------------------------------------------------------
%	SELECTED AWARDS SECTION
%----------------------------------------------------------------------------------------

\begin{rSection}{Selected Awards}
US Physics Olympiad (USAPhO) Bronze Medal + Honorable Mention \\ $3\times$American Invitational Mathematics Examination (AIME) Qualification \\ National Merit Scholar
\end{rSection}

%----------------------------------------------------------------------------------------


%----------------------------------------------------------------------------------------
%	EXAMPLE SECTION
%----------------------------------------------------------------------------------------

%\begin{rSection}{Section Name}

	%Section content\ldots

%\end{rSection}

%----------------------------------------------------------------------------------------

\end{document}
