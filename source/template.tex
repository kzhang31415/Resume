%%%%%%%%%%%%%%%%%%%%%%%%%%%%%%%%%%%%%%%%%
% Medium Length Professional CV
% LaTeX Template
% Version 3.0 (December 17, 2022)
%
% This template originates from:
% https://www.LaTeXTemplates.com
%
% Author:
% Vel (vel@latextemplates.com)
%
% Original author:
% Trey Hunner (http://www.treyhunner.com/)
%
% License:
% CC BY-NC-SA 4.0 (https://creativecommons.org/licenses/by-nc-sa/4.0/)
%
%%%%%%%%%%%%%%%%%%%%%%%%%%%%%%%%%%%%%%%%%

%----------------------------------------------------------------------------------------
%	PACKAGES AND OTHER DOCUMENT CONFIGURATIONS
%----------------------------------------------------------------------------------------

\documentclass[
	%a4paper, % Uncomment for A4 paper size (default is US letter)
	11pt, % Default font size, can use 10pt, 11pt or 12pt
]{resume} % Use the resume class

\usepackage{ebgaramond} % Use the EB Garamond font
\usepackage{hyperref}

%------------------------------------------------

\name{Kevin Zhang} % Your name to appear at the top

% You can use the \address command up to 3 times for 3 different addresses or pieces of contact information
% Any new lines (\\) you use in the \address commands will be converted to symbols, so each address will appear as a single line.

%----------------------------------------------------------------------------------------

\begin{document}

%----------------------------------------------------------------------------------------
%	EDUCATION SECTION
%----------------------------------------------------------------------------------------

\begin{rSection}{Education}
	
	\textbf{Carnegie Mellon University} \hfill \textit{June 2023 - Present} \\ 
	Bachelor of Science, Computer Science \smallskip \\
    Relevant Coursework: Principles of Imperative Computation, Mathematical Foundations for Computer Science, Great Practical Ideas in Computer Science, Matrix Theory \smallskip
 
	\textbf{Illinois Mathematics and Science Academy} \hfill \textit{August 2020 - June 2023} \\ 
	High School Diploma - GPA: 4.0/4.0 \smallskip \\
    Relevant Coursework: Computational Science, CS Seminar: Machine Learning, iOS App Development
	
\end{rSection}

%----------------------------------------------------------------------------------------
%	SKILLS SECTION
%----------------------------------------------------------------------------------------

\begin{rSection}{Skills}
	\begin{tabular}{@{} >{\bfseries}l @{\hspace{6ex}} l @{}}
		Languages & C++, JavaScript, Java, C\#, C, Swift, Python, HTML, CSS, \LaTeX, MATLAB \\
		Technologies & React, Three, Node, Express, JQuery, Linux, JSON\\
        Skills & Computational Physics, Data Analysis, Mathematics, Computer Graphics, Cryptography, \\ & Frontend Web Development, Backend Web Development, Mobile App Development
	\end{tabular}
\end{rSection}

%----------------------------------------------------------------------------------------
%	EXPERIENCE SECTION
%----------------------------------------------------------------------------------------

\begin{rSection}{Experience}

	\begin{rSubsection}{Fermilab}{June 2021 - June 2023}{Student Researcher}{Batavia, IL}
		\item Deployed machine learning models that lowered lepton jet misclassification rates by 61.8\% from the theoretical limit of cuts-based methods currently used in my group
        \item Processed LHC Run II data and Monte-Carlo simulations for training and validating the afforementioned models
	\end{rSubsection}

    \begin{rSubsection}{The Ross Mathematics Program}{June 2022 - August 2022}{Scholar}{Terre Haute, IN}
        \item Developed a simple radiosity engine (see projects section)
        \item Studied quantum-resistant encryption and signature algorithms such as the NTRU cryptosystem (NTRUEncrypt, NTRUSign, NTRU-MLS), the LLL algorithm, and the CRYSTAL algorithms (CRYSTAL-Kyber, CRYSTAL-Dilithium)
        
    \end{rSubsection}

%------------------------------------------------

\end{rSection}

\begin{rSection}{Projects}

%----------------------------------------------------------------------------------------
%	PROJECTS SECTION
%----------------------------------------------------------------------------------------
    \begin{rSubsection}{Radiosity Engine}{December 2020 - April 2021}{}{}
		\item A simple rendering engine that works by applying the finite element method to solve the rendering equation for the given scene. The radiosity method isn't view-limited and is computationally cheaper than ray-tracing for rendering the same static scene. The code and instructions for running it can be found on my GitHub page.
	\end{rSubsection}
\end{rSection}

%----------------------------------------------------------------------------------------
%	SELECTED AWARDS SECTION
%----------------------------------------------------------------------------------------

\begin{rSection}{Selected Awards and Honors}
$2\times$US Physics Olympiad (USAPhO) Bronze Medal \\ $3\times$American Invitational Mathematics Examination (AIME) Qualification \\ {[CMS]} ``Prospects for a Search for Doubly Charged Higgs Bosons at the HL-LHC". \href{https://cds.cern.ch/record/2808604?ln=en}{Publication link.} \\ National Merit Scholar
\end{rSection}

%----------------------------------------------------------------------------------------


%----------------------------------------------------------------------------------------
%	EXAMPLE SECTION
%----------------------------------------------------------------------------------------

%\begin{rSection}{Section Name}

	%Section content\ldots

%\end{rSection}

%----------------------------------------------------------------------------------------

\end{document}
